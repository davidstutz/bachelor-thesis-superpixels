\chapter{Conclusion}
\label{chapter:conclusion}

As result of this bachelor thesis, we present a new implementation of \textbf{SEEDS}, a superpixel algorithm proposed by Van den Bergh \etal \cite{VanDenBerghBoixRoigCapitaniVanGool:2012}. Additionally, we implement an extension proposed in \cite{VanDenBerghBoixRoigVanGool:2013} using pixel coordinates as smoothing term to obtain compact and regular superpixels. Furthermore, we experimented with several extensions of \textbf{SEEDS} using depth information. In conclusion, we favor two of these approaches: \textbf{SEEDS3D} using 3D point coordinates to exchange pixels between superpixels of an initial superpixel segmentation; and \textbf{SEEDS3Dn} using normal information as additional cue. As baseline we used another superpixel algorithm, especially popular because of its simplicity, called \textbf{SLIC} \cite{AchantaShajiSmithLucchiFuaSuesstrunk:2010}. In the course of the thesis we extended \textbf{SLIC} to use 3D point coordinates, as well, in order to compare its performance to \textbf{SEEDS3D}.

In the course of our experimental evaluation, we implemented most of the available measures commonly used to evaluate superpixel algorithms as extension to the popular Berkeley Segmentation Benchmark \cite{ArbelaezMaireFowlkesMalik:2011}. Among these are the Undersegmentation Error measuring the leakage of superpixels with respect to a ground truth segmentation, as well as a compactness measure proposed by Schick \etal \cite{SchickFischerStiefelhagen:2012}. Utilizing the extended Berkeley Segmentation Benchmark, we compared most of the available superpixel algorithms on the Berkeley Segmentation Dataset \cite{ArbelaezMaireFowlkesMalik:2011} and the NYU Depth Dataset \cite{SilbermanHoiemKohliFergus:2012}.

In conclusion, we summarize these experiments in three important observations. Firstly, our implementation of \textbf{SEEDS} is able to compete with all of the evaluated superpixel algorithms concerning both performance and runtime. Actually, providing near realtime, while achieving state-of-the-art performance, it is one of the fastest algorithms available (competing only with the approach proposed in \cite{FelzenswalbHuttenlocher:2004}, called \textbf{FH}, the original implementation of \textbf{SEEDS} and \textbf{SLIC}). Additionally, the implementation is easy to use: the parameters are easily interpreted and the number of superpixels can be chosen by the user and is met exactly.

Secondly, depth information is not as easily integrated into \textbf{SEEDS} as into other superpixel algorithms like \textbf{SLIC}. Nevertheless, we provide two implementations offering slightly increased performance when utilizing depth information. Based on additional experiments conducted with different algorithms using depth information, we are able to conclude that depth information does not provide a significant increase in performance. This conclusion is based on several considerations concerning the NYU Depth Dataset. To begin with, \textbf{SEEDS} does achieve excellent performance even without depth information, leaving only little room for improvement. Additionally, scenes or regions where \textbf{SEEDS} fails mostly show highly cluttered scenes with bad lighting. In these cases, the provided depth information tends to be noisy or incomplete such that using depth information in the form of 3D point coordinates or point normals does not provide sufficient information to further improve performance.
% Furthermore, using depth to adapt the size and number of superpixels may be of significance as well.% Finally, most of the superpixel algorithms are well positioned. This addresses multiple aspects as for example availability of implementations, ease of use, performance as well as runtime. In addition, the field of superpixel algorithms can be described as relatively dense, this means that for every applications a suitable superpixel algorithm can be found providing state-of-the-art performance.

Finally, many superpixel algorithms provide state-of-the-art performance. However, as these algorithms are comparable in performance, other aspects become important: the availability of implementations, ease of use, visual quality and especially runtime. Unfortunately, only few algorithms are available through popular open source computer vision libraries. Ease of use may refer to several properties as for example the provided parameters or whether the number of superpixels is controllable. In our opinion, the latter is an important property of a superpixel algorithm as the number of superpixels provides a lower bound on the achievable performance. And although a compactness measure is available \cite{SchickFischerStiefelhagen:2012}, visual quality is difficult to measure. In contrast, runtime is measurable and may be an important property, especially when considering realtime applications. Based on these considerations, we propose to use our implementation of \textbf{SEEDS} as baseline for future research as it provides all of the discussed quality characteristics.

\section{Outlook and Future Work}

As brief outlook, we see superpixel algorithms at a stage ready to be used by a great variety of computer vision applications. Actually, in our opinion, the field of superpixel algorithms appears to be saturated, that is we do not expect further research concerning novel superpixel algorithms to result in groundbreaking new approaches both increasing performance as well as decreasing runtime. However, we see a deficit concerning supervoxel algorithms. To date, there is only one algorithm ready to be used to oversegment point clouds. Furthermore, suitable benchmarks as well as annotated datasets are missing.

Future work based on our implementation of \textbf{SEEDS} includes an integration into OpenCV \cite{Bradski:2000}, one of the most popular open source computer vision libraries, to increase accessibility as well as -- if applicable -- a variant adapting the number and size of superpixels according to depth information. Additionally, due to time constraints, we were not able to conduct several interesting experiments. For example we started to implement an extension of the superpixel algorithm proposed in \cite{ConradMertzMester:2013} using depth information, however, were not able to include any experimental results. Further, we plan to use our implementation of \textbf{SEEDS} as basis for both classical segmentation as well as semantic segmentation.
% TODO: intersting experiments could not be conducted due to time constraints.